
Τα features μπορεί να μην καλύπτουν χαρακτηριστικά και της φωνής και
της μουσικής, αλλά να βασίζονται σε χαρακτηριστικά ενός από τα
δύο. Ενδιαφέρον παρουσιάζουν τα χαρακτηριστικά της ομιλίας, η οποία
λόγο των μέσων όπου την παράγουν (τα χείλη, η γλώσσα και οι φωνητικές
χορδές) έχουν περιορισμένα χαρακτηριστικά. Η μελέτη αυτών των
χαρακτηριστικών και τη χρήση τους ως features σε έναν classifier έχει
αποδειχθεί πως μπορεί να αυξήσει στην επιτυχία του διαχωρισμού
~/cite{Α}. Ενδεικτικά , πέρα από το καθιερωμένο feature των 4Hz
modulation energy λόγω του ρυθμού των συλλαβών, κάποια άλλα speech
specific features βασίζονται στην αναγνώριση του ήχου όπου παράγεται
στις φωνητικές χορδές κατά την εναλλαγή της προφοράς ενός συμφώνου σε
ένα φωνήεν ή στην μελέτη της αυτοσυσχέτησης του σήματος μετά από
φιλτράρισμα (Zero Frequency Filtered Signal) όπου παρουσιάζει
συγκεκριμένα χαρακτηριστικά μόνο στην ομιλία.

%%% Local Variables:
%%% mode: latex
%%% TeX-master: "main"
%%% End:

MIREX 2015: METHODS FOR SPEECH / MUSIC DETECTION AND CLASSIFICATION

Nikolaos Tsipas Lazaros Vrysis Charalampos Dimoulas George Papanikolaou

Αναφέρεται ότι το πρόβλημα που δόθηκε αποτελεί 2 υποπροβλήματα: Το πρόβλημα εντοπισμού δειγμάτων και το πρόβλημα κατηγοριοποίησής τους.
Για τον εντοπισμό δειγμάτων μουσικής/φωνής εφαρμόστηκε ο αλγόριθμος Random Forest σε 2 εκδοχές του : Στην πρώτη, εφαρμόστηκε μαζί με έναν Silence detection αλγόριθμο ενώ στη δεύτερη βασίστηκε μόνο στις πληροφορίες ομοιογένειας ( self similarity ? ) και στην λειτουργία του ίδιου του ταξινομητή. Επίσης, για την ταξινόμηση προτάθηκαν 2 εναλλακτικές: Στην πρώτη χρησιμοποιήθηκε ένα προ-εκπαιδευμένο μοντέλο ενώ στην δεύτερη η εκπαίδευση γίνεται κατά την αξιολόγηση των δειγμάτων.

 Χρησιμοποιήθηκαν σαν κριτήρια ( features):
- RMS ενέργεια
- ZCR ( Zero - Crossing Rate)
- Spectral rolloff ( Συχνότητα Αποκοπής ;)
- Spectral flux ( Φασματική Ροή ;)
- Spectral flatness ( Φασματική Επιπεδότητα)
- Spectral flatness per Band( Φασματική Επιπεδότητα ανά συχνοτικές ομάδες)
- MFCCs

Έπειτα χρησιμοποιήθηκε ο αλγόριθμος PCA για να μειωθούν οι διαστάσεις των διανυσμάτων κριτηρίων (feature vectors ? ) ενλω στη συνέχεια δημιουργήθηκαν οι πίνακες ομοιότητας υπολογίζοντας την ευκλίδεια απόσταση μεταξύ των δειγμάτων ήχου έτσι ώστε να χωριστούν τα τμήματα. Στη συνέχεια αυτά τα τμήματα κατηγοριοποιούνται ενώ ταυτόχρονα εφαρμόζεται ο αλγόριθμος για Silence Detection και τα δείγματα αυτά προστίθενται στα προηγούμενα. Για το πρόβλημα της κατηγοριοποίησης χρησιμοποιείται ό ίδιος αλγόριθμος Random Forest για την ταξινόμηση σε επίπεδο (frame)τμημάτων ήχου. Εφόσον για κάθε αρχείο ήχου έχουν εξαχθεί τα παραπάνω κριτήρια , κάθε τμήμα ήχου ταξινομείται στην κλάση που αποφασίζεται και έπειτα αλόκληρο το αρχείο ταξινομείται στην κλάση στην οποία ταξινομήθηκαν τα τμήματά του κατά πλειοψηφία.
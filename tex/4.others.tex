\section{Πως το έκαναν άλλοι}

Υπάρχει πληθώρα βιβλιογραφίας σχετική με το θέμα. Έχουν βρεθεί ήδη
αρκετές λύσεις, ενώ οι πιο πρόσφατες πετυχαίνουν αξιοσημείωτα
αποτελέσματα τόσο όσων αφορά την ταχύτητα του διαχωρισμού όσο και την
ακρίβεια των αποτελεσμάτων.

Πιθανές αναφορές:
\begin{itemize}[noitemsep]
\item ποιά μοντέλα (δέντρα, πιθανοτικά, neural...) είναι
  αποτελεσματικότερα με βάση τη βιβλιογραφία; Νομίζω ανάλογα με το
  paper υπάρχουν διαφορετικά αποτελέσματα σχετικά με αυτό (άλλα
  προτείνουν μπαγεσιανά και άλλα νευρωνικά) άρα παίζει ρόλο η επιλογή
  των features και στο μοντέλο, να το πούμε αυτό.. Πχ κάποια features
  έχουν μεγάλο correlation -> τα naive bayes δε τη παλεύουν σε αυτά…
  
\item ποια features χρησιμοποιούνται; Τι σημαίνει το καθένα και πως
  υπολογίζεται; Πόσο ακριβά είναι υπολογιστικά το καθένα;
  
\item ποια είναι η γενικότερη πορεία που ακολουθείται;\\
  συνήθως:
  
  \begin{enumerate}[noitemsep]
  \item παραθυροποίηση (τι τύπου; είναι επικαλυπτόμενα τα παράθυρα;
    πόσα sec είναι το καθένα;)
  \item feature extraction
  \item μετασχηματισμός του χώρου (βλέπε PCA και άλλες μεθόδους)
  \item training
  \item πρόβλεψη
  \end{enumerate}
  
\item Άρα κατά τον σχεδιασμό πρέπει εκτός από τη μέθοδο της
  παραθυροποίησης, τα features και το μοντέλο να επιλεχθούν επίσης
  κάποιος μετασχηματισμός (δεν το κάνουν πάντα) ή και άλλες
  παράμετροι. Τι άλλο preprocessing χρειάζεται;
    
\end{itemize}

Σύμφωνα με το paper ~\cite{cuckoo} το back propagation neural network πέτυχε
ακρίβεια 89.08\%, ενώ το SVM πέτυχε 90.12\% και η δική τους υλοποίηση
SVM (με τον αλγόριθμο cuckoo), CS-SVM, πέτυχε 92.75\%.

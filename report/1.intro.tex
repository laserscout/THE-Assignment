\section{Εισαγωγή}

Το ζητούμενο της εργασίας είναι η ανάπτυξη ενός μοντέλου μηχανικής
μάθησης το οποίο, παρέχοντας ένα αρχείο ήχου, θα μπορεί να ξεχωρίσει
ανάμεσα στα κομμάτια του χρόνου που περιέχουν ομιλία (speech) και
μουσική (music), όπως παρουσιάζεται στον διαγωνισμό MIREX 2018:Music and/or Speech Detection 
\footnote{https://www.music-ir.org/mirex/wiki/2018:Music\_and/or\_Speech\_Detection} .
Η εργασία επικεντρώνεται στην εύρεση των δειγμάτων που περιέχουν είτε ομιλία είτε μουσική
και στην ταξινόμησή τους.

Πρόκειται για ένα δυαδικό πρόβλημα ταξινόμησης που είναι σημαντικό καθώς έχει
εφαρμογές σε πλατφόρμες κοινωνικών δικτύων για την αναγνώριση
περιεχομένου με πνευματικά δικαιώματά, σε συστήματα αυτόματης
αναγνώρισης διαφημίσεων, μοντέρνα "έξυπνα" βοηθητικά ακοής κ.α. Η
πρόσφατη βιβλιογραφία περιέχει θεματολογία όπου στοχεύει είτε στην
ανάπτυξή αλγορίθμων για γρήγορη και φθηνή υπολογιστικά ταξινόμηση,
είτε στην αναγνώριση πολύ μεγάλης ακρίβειας. Αυτό διότι αυτή τη
στιγμή η αναγνώριση με ποσοστό επιτυχίας γύρω στο 98\% είναι κάτι
συνηθισμένο.